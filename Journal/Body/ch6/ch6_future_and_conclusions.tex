\chapter{GENERAL CONCLUSION}
\label{future-work}


This is the opening paragraph to my thesis which
explains in general terms the concepts and hypothesis
which will be used in my thesis.

With more general information given here than really
necessary.

\section{Summary And Discussion}

Here initial concepts and conditions are explained and
several hypothesis are mentioned in brief.


\subsection{Hypothesis}

Here one particular hypothesis is explained in depth
and is examined in the light of current literature.

As can be seen in Table~\ref{nothingelse} it is
truly obvious what I am saying is true.

\begin{sidewaystable} \centering
\isucaption{This table shows almost nothing but is a
sideways table and takes up a whole page by itself}
\label{nothingelse}
% Use: \begin{tabular{|lcc|} to put table in a box
\begin{tabular}{lcc} \hline
\textbf{Element} & \textbf{Control} & \textbf{Experimental} \\ \hline
Moon Rings & 1.23 & 3.38 \\
Moon Tides & 2.26 & 3.12 \\
Moon Walk & 3.33 & 9.29 \\ \hline
\end{tabular}
\end{sidewaystable}

\subsubsection{Parts of the hypothesis}

Here one particular part of the hypothesis that is 
currently being explained is examined and particular
elements of that part are given careful scrutiny. \cite{chenGraphHomotopyGraham2001}, \cite{chenGraphHomotopyGraham2001},\cite{virkContractibilityRipsComplexes2024}
Here is an equation \[x^2 + y^2 = 8.\]

% Below \subsubsection
% Sectional commands: \paragraph and \subparagraph may also be used


%\section{Criteria Review}

%Here certain criteria are explained thus eventually
%leading to a foregone conclusion.

%\bibliographystyle{acm} % use for numbered citations along with options given in preamble. Look at the main thesis.tex file
\printbibliography[heading=subbibnumbered]

% \section{Bibliography}

% \bibliographystyle{apa}
% % \vspace{-20pt}
% \begingroup
%     \setlength{\bibsep}{13.2pt}
%     \linespread{1}\selectfont
%     \bibliography{master_bib}
% \endgroup
% \clearpage
% \pagebreak
